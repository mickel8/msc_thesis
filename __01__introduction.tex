\section{Introduction}
\label{sec:introduction}

An interactive communication is bidirectional data exchange between people or people and machines.
It consists of two or more participants and has to be responsive meaning the result of performed action is delivered almost immediately.
In many cases interactive communication has to be also secure and reliable.
Examples of interactive communication are controlling remote robot in tele-surgery, doing exercises in tele-rehabilitation and
some types of multimedia sessions e.g.\ video conference, collaborative work or chat.
On the other hand to interactive communication does not belong exchanging emails (time between subsequent messages is too long) or
video on demand and streaming (they are not bidirectional).

Years of researches, protocol improvements and network devices optimizations have made interactive communication really fast.
We can talk with another person almost in the real time with a delay of several milliseconds.
Having said that it is perfectly correct to ask a question whether communication can be even faster, more effective and reliable and whether it is still worth to make improvements while we already have well prospering ecosystem.
In 2006 Prof.\ Edward Delp said:
\begin{displayquote}
    Is video coding dead?
    Some feel that, with the higher coding efficiency of the H.264/MPEG-4 Advanced Video Coding (AVC) standard (2x as compared to MPEG-2), perhaps there is not much more to do.
    I must admit that I have heard this \textquote{compression is dead} argument at least four times since I started working in image and video coding in 1976.\cite{4015574}
\end{displayquote}
Despite the fact that we are able to achieve very good performance in video conferencing systems there is still much to do.
Serious gaming, tele-surgery and tele-rehabilitation are all subject to tactile internet in which reliability, security and low latency are fundamental.
In such cases delay of 1 ms is desirable~\cite{the-tactile-internet}.

On the other hand, complexity of current standards makes them hard to implement, monitor and maintain.
One of them is WebRTC which uses over 10 different protocols and leaves implementation of signalling plane to the user.

QUIC is a new transport protocol standardized in RFC 9000 on May in 2021 and provides interesting and promising set of features.
%Its integration with TLS 1.3 significantly reduces connection establishment time while a stream multiplexing in a single connection resolves head of line blocking problem of HTTP/2.
%QUIC introduces also special connection identifiers that prevents from additional handshakes in case of network switching and defines own flow control and congestion control mechanisms that are based on TCP ones.
%QUIC packets are encapsulated in UDP datagrams which makes QUIC user level protocol i.e.\ there is no need to modify kernel implementation to provide operating system with support for QUIC\@.
%
It is mostly deployed in conjunction with a new version of HTTP called HTTP/3 and is intended to replace TCP and HTTP/2.
Therefore most publications are focused on web domain comparing QUIC and HTTP/3 with TCP/TLS and HTTP/2.

However, it turns out that a lot of QUIC features mainly stream multiplexing, pluggable congestion control, integration with TLS 1.3
or enhanced connection establishment mechanism can be useful in other domains.
As a result a number of IETF drafts emerged to expand QUIC to new areas.
The most important one in terms of interactive communication is \textit{An Unreliable Datagram Extension to QUIC}.
It allows for combining reliable and unreliable data exchange in the same QUIC connection.

This thesis tries to answer a question whether QUIC can introduce a significant improvement in the interactive communication.
To this end, I evaluate QUIC in four domains.
The first one is ease of use and deployment.
Here I am describing current status of QUIC implementations and I am presenting how to create a Hello World application.
The second one is encryption process which is mandatory in QUIC\@.
I am analyzing its performance overhead comparing to the encryption process in SRTP protocol.
The next one is media data transmission over QUIC\@.
Here I am testing DATAGRAM extension to QUIC and I am presenting my simple application that uses this extension to send H264 encoded video file over QUIC and Membrane Framework.
The last domain is related to WebRTC\@.
I am presenting how QUIC can simplify its protocol stack.

The structure of this document is as follows.
Section~\ref{sec:quic-overview} introduces basic concepts of QUIC and compares QUIC to other transport protocols.
Section~\ref{sec:state-of-the-art} presents state of the art.
Section~\ref{sec:evaluation} addresses problems described above.
Section~\ref{sec:conclusions} includes conclusions and further work. 

