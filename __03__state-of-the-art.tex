\section{State of the art}
\label{sec:state-of-the-art}

The number of QUIC publications at the moment of writing this thesis is limited.
The main reason of such situation is that QUIC was standardize just a few months ago.
In this section I present an overview of publications that are important in terms of current main QUIC usage which is
HTTP/3, are somehow related to transporting media over QUIC or are dedicated to aspects that are also important
in interactive communication.

\textit{The QUIC Fix for Optimal Video Streaming} article~\cite{the-quic-fix-for-optimal-video-streaming} introduces unreliable data transmission over QUIC and presents how combination of reliable and unreliable transmission fares in video streaming and outperforms TCP and reliable mode of QUIC\@.
Authors of this document tag H.264 video frames depending on their importance.
\textit{I-Frames} which are independent frames meaning they can be displayed without any additional frames are marked to be sent reliably.
\textit{P-Frames} which are much smaller in size and code only the difference between \textit{I-Frame} and the next frame are marked to be sent unreliably.
The same approach is applied to \textit{B-Frames} that depends on both \textit{I-Frames} and \textit{P-frames}.
Authors state that the loss of \textit{P-Frames} or \textit{B-Frames} has minimal or no impact on the user's quality of experience (QoE).
They use i.a. \textit{buffering ratio (bufRatio)} and \textit{rate of buffering (rateBuf)} metrics to measure the difference in performance of streaming video over codec-agnostic DASH protocol used with TCP, traditional QUIC and QUIC with addition of unreliable transmission.
\textit{BufRatio} is the amount of time spent on re-buffering video comparing to the total video duration, represented as a percentage.
\textit{RateBuf} is the frequency of the buffering events~\cite{impact-of-video-quality-on-user-engagement}.
As a result, \textit{bufRatio}, with packet loss set to 0.64\%, for TCP is 105\%, for QUIC is 30\% and for QUIC with addition of unreliable data transmission is less than 1\%.
\textit{RateBuf}, with the same packet loss, is equal to 50\% for TCP, 19\% for QUIC and nearly 0\% for QUIC with addition of unreliable data transmission.

\textit{QUIC: Better for what and for whom?} article~\cite{quic-better-for-what-and-for-whom} compares the page load time (PLT) for HTTP/2 requests over QUIC and TCP/TLS in different network conditions and architectures as well as for different website complexities.
Authors prepared both local and remote testbeds.
In the remote one client's machine is connected to the Internet over ADSL (to router over Ethernet or Wi-Fi) or 4G\@.
Different network conditions mean different packet loss rate and delay.
In case of complexity of site there is Youtube service in which different resources might be distributed over many servers and Doctor (website of the ANR project) website where ale files are located on the same server.
Authors concludes that QUIC outperforms HTTP/2 over TCP/TLS in unstable networks but in case of stable and reliable networks the benefits of QUIC are not so obvious.

\textit{Game of protocols: Is QUIC ready for prime time streaming?} article~\cite{game-of-protocols} compares QUIC and TCP in HAS (HTTP adaptive streaming) applications.
To this end, authors performed experiments in four scenarios.
Frame-seek scenario is about seeking to the specified frame in the video.
Connection-switch scenario is about changing the connection from e.g.\ Wi-Fi to 3G\@.
Multiplexing scenario is about comparing different stream multiplexing techniques i.e.\ HTTP/1.1 over varying number of TCP connections, HTTP/2 with parallel requests over a single TCP connection and QUIC over a single UDP connection.
Fairness scenario is about checking fairness of congestion control mechanisms while there are multiple competing clients.
Authors also used three different adaptive algorithms: BASIC, SARA and BBA-2.
In the frame-seek scenario QUIC behaved better for all adaptive algorithms and types of networks reducing numerically \textit{rebufferRate} metric (which is a similar metric to \textit{bufRatio}) by 1--3\%.
In the WiFi-LTE and WiFi-3G connection-switch scenario, QUIC also behaved better for all adaptive algorithms reducing numerically \textit{rebufferRate} by 1\% for SARA and BBA-2 and by 7\% for BASIC\@.
Evaluation of different multiplexing techniques showed that in the typical network conditions all three methods resulted in the similar average playback bitrate.
In the large delay and typical loss network, QUIC performed best.
For typical delay and large loss as well as for large delay and large loss, HTTP/1.1 with varying number of TCP connections was better than QUIC\@.
HTTP/2 over single TCP connection didn't manage to beat any of the other multiplexing techniques in any of the network conditions.
Fairness examination of congestion control mechanisms in TCP and QUIC showed that for typical packet loss and delay both protocols guarantee fair resource access for competing clients which results in the similar average playback bitrate.
In the large loss scenario single QUIC client was able to achieve higher bitrate (about 37\%) than single TCP client.
For competing clients, QUIC still was able to achieve better bitrate than TCP (about 40\%).
In the large delay scenario, the results were similar to the large loss scenario.
The last one scenario with large loss and large delay showed that TCP client was able to achieve higher bitrate (about 16\%) both when competing and not competing with QUIC client.

\textit{And QUIC meets IoT: performance assessment of MQTT over QUIC} article~\cite{and-quic-meets-iot} maps MQTT to QUIC
and compares its performance with state-of-the-art MQTT over TCP\@.
Authors performed two tests.
In the first one, they send 1000 MQTT messages to the broker which then are sent to the subscriber.
In the second one, they establish MQTT connection and send only one message and close the connection.
The purpose of the second scenario was to see improvements introduced by QUIC in terms of faster connection establishment.
Both tests were performed in three different, emulated network environments: Wi-Fi, 4G/LTE and Satellite.
Authors states that MQTT over QUIC outperforms MQTT over TCP however, they expected a bigger advantage
of QUIC in the second scenario due to QUIC 0-RTT packets.

\textit{Emerging Interactive Applications over QUIC} article~\cite{9045245} compares QUIC performance with TCP and UDP
using three multimedia services.
The first service which is cloud gaming represents an interactive communication.
The second and third ones which are 4K video streaming and client-server online gaming represent classic multimedia
applications.
Each client in each service was connected to the Access Point which had access to the Internet.
The bottle neck was the connection between AP and router to which three servers (one for each service) were connected.
The first experiment tests QUIC fairness and shows that average achieved bandwidth by Cloud Gaming and 4K streaming
is more balanced when using QUIC\@.
The second experiment tests QUIC end-to-end latency and shows a significant delay reduction for all three services when using QUIC\@.
In the last scenario, authors compare Round Trip Time when streaming 4K video using QUIC and TCP\@.
Experiment shows that RTT for QUIC is lower than for TCP\@.
To sum up in all scenarios QUIC outperforms traditional transport protocols both for interactive and classic multimedia
applications.

\textit{Media QoE Enhancement With QUIC} article~\cite{7562075} uses QUIC stream prioritization to improve Quality of Experience (QoE).
Authors focus on MPEG-DASH based application.
They prioritize streams that carry multimedia data depending on the client's player buffer.
If its level is below 20 sec then client requests subsequent media chunks with high priority.
For buffer level between 20 and 80 sec medium priority is requested and for buffer level above 80 sec high priority.
The experiment shows that stream prioritization gain in Initial Buffering Time can be from 6\% to 49\% depending on
bandwidth limit and video resolution.
%
%\textit{Impact of ACK Scaling Policies on QUIC Performance} article~\cite{9524947}
%
%
%
%\textit{An attempt at introducing Multipath in QUIC} article~\cite{8806051}

%\textit{A Pure HTTP/3 Alternative to MQTT-over-QUIC in Resource-Constrained IoT} article~\cite{iot} compares performance
%of public-subscribe architecture realized with HTTP/3 over QUIC and MQTT over QUIC\@.
